\newcommand{\docNome}{ Relazione Progetto di Tecnologie Web}
\newcommand{\titoloProgetto}{ Job Finder}
\newcommand{\Tber}{Tommaso Berlaffa } 
\newcommand{\MatT}{1201234}
\newcommand{\Mspa}{Milo Spadotto } 
\newcommand{\MatM}{1122180}
\newcommand{\Plau}{Pietro Lauriola } 
\newcommand{\MatP}{1224820}
\newcommand{\Amat}{Alberto Materazzo} 
\newcommand{\MatA}{1144597}

\documentclass[12pt, a4paper,table]{article}
\usepackage[utf8]{inputenc}
\usepackage{fancyhdr}
\usepackage{geometry}
\usepackage{xcolor}
\usepackage{array}
\usepackage{graphicx}
\usepackage{hyperref}
\usepackage[document]{ragged2e}
\usepackage[export]{adjustbox}
\hypersetup{
	pdfborder = {0 0 0}
}
\geometry{a4paper,top=3cm,bottom=3cm,left=2cm,right=2cm}
\title{\textsc{\docNome}}

\date{}
\author{}
\definecolor{footer-gray}{HTML}{808080}
\pagestyle{fancy}
\fancyhf{}
\fancyhead{}

\rhead{\textcolor{footer-gray}{\docNome} }
\lhead{\textcolor{footer-gray}{Job Finder}}
\setlength{\headheight}{14.49998pt}

\fancyfoot{}
\cfoot{\textcolor{footer-gray}{Pagina \thepage } }

\begin{document}
	\maketitle
	\begin{center}
    \begin{figure}[h]
      \includegraphics[scale=0.095]{Images/Logo_Università_Padova.svg.png}
      \hspace{21em}
      \includegraphics[scale=0.095]{Images/JF_1200.png}
    \end{figure}
    Titolo Progetto : \huge{\titoloProgetto}
		\\
		\vspace{1em}
		\normalsize Email Referente :  \href{mailto:tommaso.berlaffa@studenti.unipd.it}{tommaso.berlaffa@studenti.unipd.it}\\
		
		\vspace{2em}
		\textbf{Componenti del Gruppo :}
		\vspace{0.5em}
		\\ \Tber : \MatT
		\vspace{0.5em}
		\\ \Mspa : \MatM
		\vspace{0.5em}
		\\ \Plau : \MatP
		\vspace{0.5em}
		\\ \Amat : \MatA
    \vspace{0.5em}
    \\ Credenziali :
    \begin{center}
      \begin{tabular}{ |c|c|c| } 
      \hline
      & Nickname & Password \\
      \hline
      User Generico & user & user\\ 
      \hline
      Admin & admin & admin \\ 
      \hline
      \end{tabular}
    \end{center}
    \vspace{0.5em}
    Link al progetto : 

    
	\end{center}

  \newpage
  \tableofcontents
  ragged-left ( \raggedright )
	\newpage
	\section{Introduzione}
	\subsection{Abstract}
	Il sito \textbf{Job Finder} si pone come intermediario tra utenti datori di lavoro e utenti lavoratori.\\
  I tipi di lavoro svolti su questo sito saranno esclusivamente e totalmente di tipo \textit{Online}. \\
	Gli utenti dovranno registrarsi per ottenere diversi privilegi, quali cercare lavoro o creare offerte di lavoro, a cui altri utenti potranno rispondere tramite una offerta, 
  chiamata \textit{bid}.\\
  Ogni utente, dopo aver effettuato il login, avrà accesso ad un'area riservata, nella quale potrà visualizzare e modificare i propri dati personali. In quest'area sarò inoltre 
  possibile visualizzare le proprie offerte $($passate e correnti$)$, i lavori svolti e le offerte a cui si sta attualmente candidando.\\
  Non esisterà una divisione tra Utenti che offrono lavoro e utenti che si propongono, ma ci saranno solamente utenti "normali" ed Admin. Abbiamo optato per questa scelta poiché 
  uno stesso utente si potrebbe trovare a svolgere un lavoro, ma commissionandone un altro allo stesso tempo.\\
  L'intero sito utilizzerà la lingua inglese come unica lingua disponibile, poiché limitare le possibili offerte di lavoro in base a lingua o nazionalità limiterebbe solamente 
  il sito.\\
  Viene garantita l’accessibilità al sito, rendendo il sito navigabile a chiunque.\\
  L’usabilità risulta essere un altro tema di particolare importanza: vengono infatti rispettate le linee guida fornite da W3C e viene fatto in modo che la navigazione sul sito
  risulti essere la più fluida possibile, fornendo sempre aiuti per fare in modo che l’utente non si senta perso e sappia dov'é e dove può andare.
	
	
	\newpage
	\section{Analisi}
	\subsection{Analisi di Utenza}
	
	Il sito Job finder si rivolge ad un tipo di pubblico esperto, già introdotto al mondo dell’informatica. Viene fatto in modo che anche un utente non esperto e alle prime armi riesca a trovare tutte le informazioni di cui è alla ricerca senza troppe difficoltà.
	\\	L’utenza di questo sito si quindi può dividere in 3 macrocategorie :
	\begin{itemize}
		\item 	Utente nuovo ed inesperto : arriva al sito per la prima volta, senza conoscerne alcun dettaglio. Per questo tipo di utente, la pagina di Home deve contenere tutte le informazione necessarie a risolvere qualsiasi dubbio riguardo allo scopo e alle funzionalità del sito. È inoltre necessario fare in modo che questo nuovo utente non si senta soverchiato dalle informazioni del sito, per questo motivo viene utilizzato un linguaggio di facile comprensione
		\item Utente in cerca di Lavoro : utente che conosce il sito e vuole sfruttare le proprie skills per ottenere un lavoro. Le zone di ricerca di lavoro e di UserProfile sono quelle dove questo tipo di utente passerà la maggior parte di tempo della navigazione del sito
		\item Utente in cerca di Lavoratori : utente che conosce il sito e necessita di un Lavoratore in grado di svolgere un compito ben specifico. Le zone di creazione lavoro e UserProfile sono quelle dove questo tipo di utente svolgerà la maggior parte delle proprie azioni
	\end{itemize}
	Il tipo di utente, indipendentemente dal motivo per cui si trova sul sito, viene diviso in 3 categorie :
	\begin{itemize} 
		\item l’user che non ha ancora fatto il Log In o la Registrazione (User Non Loggato)
		\item l’user che ha effettuato il Log In (User Generico) 
		\item l’utente con privilegi (Amministratore)
	\end{itemize}
	Effettuare il Log In permette ad un Utente la visualizzazione dello User Profile e della Creazione di una offerta di Lavoro. Inoltre, accederà alla possibilità di candidarsi ad un offerta di Lavoro.
	Un utente Amministratore avrà ulteriori privilegi, quali i ban su Utenti e l’eliminazione di offerte di lavoro.
	Inoltre, il sito è stato realizzato totalmente in lingua inglese ponendosi verso un suo utilizzo a livello globale. 
	\\Questa scelta è stata fatta poiché i lavori vengono eseguiti completamente Online, rendendo così una localizzazione dei lavori solamente dannosa per gli utenti. (minore possibilità di trovare un lavoro o un lavoratore poiché limitati dalla propria posizione geografica)
	\subsection{Analisi Requisiti} 
	Il sito si occupa principalmente di soddisfare 2 richieste :
	\begin{itemize}
		\item Ricerca di un lavoro, tramite le pagine di visualizzazione dei lavori disponibili, a cui un utente interessato può fare una proposta
		\item Ricerca di un lavoratore, tramite la creazione di un’offerta di lavoro, al quale il possibile candidato dovrà proporsi
	\end{itemize}
	Oltre a questi due requisiti, su cui si basa l’intero sito, sono presenti altre funzionalità di interesse, quali :
	\begin{itemize}
		\item Descrizione delle funzionalità del sito
		\item Sign Up, Log In, Log Out. La visualizzazione di alcune pagine, come lo User Profile o la Creazione di una Offerta di Lavoro, non sono visibili agli utenti non attualmente Loggati.
		\item Cronologia dei Lavori Svolti (sia da parte del creatore dell’Offerta, sia da parte del Lavoratore), Cronologia delle Offerte e Offerte attualmente in atto.
		\item Possibilità di filtrare le varie offerte di Lavoro
		\item Possibilità di Visualizzare informazioni su Utenti che si propongono per un offerta di lavoro
		\item Lettura e Scrittura di Feedback sul Lavoro svolto
	\end{itemize}
		
	\section{Progettazione}
	
  \subsection{Design}
  Per il design del sito, sono state perseguite due principali strade :
  \begin{itemize}
    \item \textit{Desktop First}: il nostro sito si rivolge ad un utenza molto tecnica e pensiamo che l'utilizzo del sito avverrà principalmente da Desktop. Viene comunque data importanza all'utenza che deciderà di approcciarsi a questo sito tramite smartphone:
    \item \textit{Layout a 4 Pannelli}: utilizzato in tutte le pagine del sito tranne UserProfile, dove viene utilizzato un Layout a 5 Pannelli,aggiungendo una Sidebar per la navigazione tra gli elementi di userprofile.
  \end{itemize}
  
  I 4 pannelli che compongono il layout delle nostre pagine sono:
  \begin{enumerate}
    \item \textbf{Header}: composto dal logo del sito e dai link alle altre pagine;
    \item \textbf{Breadcrumb}: contiene i Link alle pagine "precedenti" a quella che si sta attualmente visitando; il breadcrumb è di vitale importanza per far sì che l'utente
      non si senta perso nella navigazione del sito. Nel caso del nostro sito, il breadcrump è stato scritto in maniera da simulare un path di un filesystem;
    \item \textbf{Main}: corrisponde al vero e proprio contenuto della pagina;
    \item \textbf{Footer}: contiene informazione su come contattare l'Admin in caso di problemi e il link per le FAQ.
  \end{enumerate}

  Il quinto pannello, overro la \textbf{sidebar}, è utilizzato nella pagina UserProfile. 
  Questa pagina raccoglie informazioni utili dell'utente (informazione su utente, sui lavori svolti e proposti) e la possibilità di modificare i dati utente.


  \subsection{Database}




  \subsection{Obiettivi}
  
	
	\section{Presentazione}

  \subsection{Desktop}

  \subsection{Mobile}

  \subsection{Print}
	
	\section{Implementazione}
	

  \subsection{HTML}


  \subsection{CSS}


  \subsection{JavaScript}


  \subsection{PHP}


  \subsection{MySql}
	
	\section{Validazione}
La validazione risulta essere uno dei punti crucialie del progetto. Qyesto perché permette di verificare che siano stati rispettati gli standard W3C per quanto riguarda HTML e CSS.\\
Per validare il sito sono stati utilizzati i seguenti strumenti:
\begin{itemize}
	\item \textbf{W3C HTML Validator}: è un tool online che, dato in input un file .html o del codice HTML, effettua la validazione del codice HTML al fine di assicurare la qualità della pagina.
	\item \textbf{W3C CSS Validator}: è un tool online che, dato in input un file .css o del codice CSS, effettua la validazione del codice CSS, permettendo così di assicurare la validità del codice sottoposto.
\end{itemize}
	
	\section{Test}

  Gli strumenti utilizzati per il Testing generale del sito sono stati:
  \begin{itemize}
    \item \textit{WAVE}, ovvero uno strumento utile a rendere il proprio sito più accessibile ad utenti con disabilità. Questo strumento infatti può identificare diversi errori di accessbilità (come quelli presenti nelle Guideline di WCAG). I test di WAVE sono stati fatti tramite estensioni del browser Chrome.
    \item \textit{Silktide}, un'estensione web che permette di testare la propria pagina andando a simulare diverse disabilità, tra cui (per citarne alcune) dislessia e daltonismo. Questo tipo di software ci permette quindi di rendere le nostre pagine più facilmente esplorabili da questa tipologie di utenza.
  \end{itemize}
  Sono stati inoltre utilizzati strumenti più specifici per altri tipi di testing.
	
  \subsection{Testing di Compatibilità}
    Il testing delle pagine è stato effettuato su diversi browsers, ovvero :
    \begin{itemize}
      \item Google Chrome, su cui è stato testato principalmente il sito;
      \item Mozilla Firefox;
      \item  Microsoft Edge;
      \item Safari.
    \end{itemize}
    Abbiamo deciso di tralasciare Internet Explorer poiché quest'ultimo risulta non essere più supportato.

  \subsection{Testing del contrasto dei Colori} 
    WAVE permette, in maniera molto semplice, di controllare se nel sito sono presenti contrasti a livello cromatico che potrebbero risultare fastidiosi ad alcune categorie di utenti.
    Inoltre, è stato anche utilizzato il \href{https://webaim.org/resources/contrastchecker/}{Contrast Checker} fornito da WebAim per assicurarci che i colori da noi scelti risultassero non fastidiosi all'utente.

  \subsection{Testing di Grandezza delle Pagine}
    Una caratteristica molto importante di un buon sito web è la leggerezza, infatti nella creazione di un sito web è buona pratica fare in modo che le pagine non risultino essere troppo pesanti. Questo potrebbe rendere il rendering delle pagine lento e creare una sensazione negativa nell'utente.\\
    Per questo motivo, viene consigliato di fare in modo che il preso delle pagine sia il minore possibile, generalmente mantenendolo tra i 2 ed 1 Megabyte. 

  \subsection{Testing di Altezza delle Pagine}
    Vengono disposte pratiche per fare in modo che non vi siano pagine "troppo alte" e, in caso questo succeda comunque, vengono predisposti meccanismi per tornare indietro. \\
    Esempi di questi meccanismi possono essere :
    \begin{enumerate}
      \item Nella pagina \textit{Findjob}, i lavori posso idealmente essere migliaia ma ne verranno solamente visualizzati un massimo di 10 per volta. Questo avviene grazie alla divisione del contenuto in sottopagine;
      \item Nelle pagine \textit{UserProfile} di \textit{Your Job Offers} e \textit{Your Bids}, le tabelle vengono riempite dinamicamente con i dati presenti nel database. Per questo motivo, non conoscendo la Grandezza
      delle tabelle, vengono creati dei link di "go back to the top" per far sì che, nel caso la tabella generata risulti essere troppo grande, l'utente abbia comunque la capacità di tornare all'inizio della pagina senza alcuna difficoltà.
    \end{enumerate}

  \input{Capitoli/07_Lavoro.tex}

\end{document}