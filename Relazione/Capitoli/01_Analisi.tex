\section{Analisi}
	
  \subsection{Analisi di Utenza}	
	  Il sito Job finder si rivolge ad un tipo di pubblico esperto, già introdotto al mondo dell’informatica. Viene fatto in modo che anche un utente non esperto e alle prime 
    armi riesca a trovare tutte le informazioni di cui è alla ricerca, senza troppe difficoltà.\\	
    L’utenza di questo sito si quindi può dividere in 3 tipologie:
	  \begin{itemize}
      \item \textbf{Utente nuovo ed inesperto}: arriva sul sito per la prima volta, senza conoscerne alcun dettaglio. Per questo tipo di utente, la pagina di Home e le FAQ devono contenere tutte 
      le informazione necessarie a risolvere qualsiasi dubbio riguardo allo scopo e alle funzionalità del sito. È inoltre necessario fare in modo che questo nuovo utente non si
      senta soverchiato dalle informazioni del sito, per questo motivo viene utilizzato un linguaggio di facile comprensione e vengono mostrare a schermo solamente le 
      informazioni necessarie;
      \item \textbf{Utente in cerca di Lavoro}: utente che conosce il sito e vuole sfruttare le proprie conoscenze ed abilità per ottenere un lavoro. Le zone di ricerca di lavoro e di profilo utente 
      sono quelle dove questo tipo di utente passerà la maggior parte di tempo della navigazione del sito;
      \item \textbf{Utente in cerca di Lavoratori}: utente che conosce il sito e necessita di un lavoratore in grado di svolgere un compito ben specifico. Le zone di creazione lavoro 
      e profilo utente sono quelle dove questo tipo di utente svolgerà la maggior parte delle proprie azioni.
    \end{itemize}
    Il tipo di utente, indipendentemente dal motivo per cui si trova sul sito, viene inoltre diviso in 3 categorie:
    \begin{itemize} 
      \item l’user che non ha ancora fatto il Log In o la Registrazione (\textit{User Non Loggato})
      \item l’user che ha effettuato il Log In (\textit{User Generico}) 
      \item l’utente che ha effettuato il Log In e possiede privilegi (\textit{Amministratore})
    \end{itemize}

  \subsection{Casi d'Uso}
    Alle 3 tipologie di utenza precedente elencate saranno accessibili diverse pagine:
    \begin{itemize}
      \item a tutti gli utenti saranno visibili le pagine Home, FindJob, Login, SignUp e FAQ;
      \item a tutti gli utenti che hanno effettuato il login saranno visibili tutte le pagine (comprendono CreateJob, UserProfile e ViewJob/ViewUser);
      \item agli utenti Amministratori saranno visualizzabili ulteriori elementi nelle pagine che permettono modifiche a dati/informazioni di altri utenti, in aggiunta ad una pagina per cercare specifici utenti e lavori.
    \end{itemize}

  \subsection{Analisi Requisiti} 
    Il sito si occupa principalmente di soddisfare 2 richieste:
    \begin{itemize}
      \item \textbf{Ricerca di un lavoro}, tramite le pagine di visualizzazione dei lavori disponibili, a cui un utente interessato può fare una proposta;
      \item \textbf{Ricerca di un lavoratore}, tramite la creazione di un’offerta di lavoro, al quale il possibile candidato potrà proporsi.
    \end{itemize}
    Oltre a questi due requisiti, su cui si basa l’intero sito, sono presenti altre funzionalità di interesse, quali:
    \begin{itemize}
      \item Descrizione delle funzionalità del sito;
      \item Sign Up, Login, Logout. La visualizzazione di alcune pagine, come lo User Profile o la creazione di un'offerta di Lavoro, non sono visibili agli utenti non 
      attualmente loggati;
      \item Cronologia dei lavori svolti e creati, e delle offerte passate ed attualmente in atto;
      \item Ricerca di offerta di lavoro con possibilità di filtrare le varie offerte;
      \item Visualizzazione informazioni su Utenti che si propongono per un offerta di lavoro;
      \item Possibilità, da parte di un datore di Lavoro, di dare un Feedback sul lavoro svolto.
    \end{itemize}
