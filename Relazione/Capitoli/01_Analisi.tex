\section{Analisi}
	\subsection{Analisi di Utenza}
	
	Il sito Job finder si rivolge ad un tipo di pubblico esperto, già introdotto al mondo dell’informatica. Viene fatto in modo che anche un utente non esperto e alle prime armi riesca a trovare tutte le informazioni di cui è alla ricerca senza troppe difficoltà.
	\\	L’utenza di questo sito si quindi può dividere in 3 macrocategorie :
	\begin{itemize}
		\item 	Utente nuovo ed inesperto : arriva al sito per la prima volta, senza conoscerne alcun dettaglio. Per questo tipo di utente, la pagina di Home deve contenere tutte le informazione necessarie a risolvere qualsiasi dubbio riguardo allo scopo e alle funzionalità del sito. È inoltre necessario fare in modo che questo nuovo utente non si senta soverchiato dalle informazioni del sito, per questo motivo viene utilizzato un linguaggio di facile comprensione
		\item Utente in cerca di Lavoro : utente che conosce il sito e vuole sfruttare le proprie skills per ottenere un lavoro. Le zone di ricerca di lavoro e di UserProfile sono quelle dove questo tipo di utente passerà la maggior parte di tempo della navigazione del sito
		\item Utente in cerca di Lavoratori : utente che conosce il sito e necessita di un Lavoratore in grado di svolgere un compito ben specifico. Le zone di creazione lavoro e UserProfile sono quelle dove questo tipo di utente svolgerà la maggior parte delle proprie azioni
	\end{itemize}
	Il tipo di utente, indipendentemente dal motivo per cui si trova sul sito, viene diviso in 3 categorie :
	\begin{itemize} 
		\item l’user che non ha ancora fatto il Log In o la Registrazione (User Non Loggato)
		\item l’user che ha effettuato il Log In (User Generico) 
		\item l’utente con privilegi (Amministratore)
	\end{itemize}
	Effettuare il Log In permette ad un Utente la visualizzazione dello User Profile e della Creazione di una offerta di Lavoro. Inoltre, accederà alla possibilità di candidarsi ad un offerta di Lavoro.
	Un utente Amministratore avrà ulteriori privilegi, quali i ban su Utenti e l’eliminazione di offerte di lavoro.
	Inoltre, il sito è stato realizzato totalmente in lingua inglese ponendosi verso un suo utilizzo a livello globale. 
	\\Questa scelta è stata fatta poiché i lavori vengono eseguiti completamente Online, rendendo così una localizzazione dei lavori solamente dannosa per gli utenti. (minore possibilità di trovare un lavoro o un lavoratore poiché limitati dalla propria posizione geografica)
	\subsection{Analisi Requisiti} 
	Il sito si occupa principalmente di soddisfare 2 richieste :
	\begin{itemize}
		\item Ricerca di un lavoro, tramite le pagine di visualizzazione dei lavori disponibili, a cui un utente interessato può fare una proposta
		\item Ricerca di un lavoratore, tramite la creazione di un’offerta di lavoro, al quale il possibile candidato dovrà proporsi
	\end{itemize}
	Oltre a questi due requisiti, su cui si basa l’intero sito, sono presenti altre funzionalità di interesse, quali :
	\begin{itemize}
		\item Descrizione delle funzionalità del sito
		\item Sign Up, Log In, Log Out. La visualizzazione di alcune pagine, come lo User Profile o la Creazione di una Offerta di Lavoro, non sono visibili agli utenti non attualmente Loggati.
		\item Cronologia dei Lavori Svolti (sia da parte del creatore dell’Offerta, sia da parte del Lavoratore), Cronologia delle Offerte e Offerte attualmente in atto.
		\item Possibilità di filtrare le varie offerte di Lavoro
		\item Possibilità di Visualizzare informazioni su Utenti che si propongono per un offerta di lavoro
		\item Lettura e Scrittura di Feedback sul Lavoro svolto
	\end{itemize}