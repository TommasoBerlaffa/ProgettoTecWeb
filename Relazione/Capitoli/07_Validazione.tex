\section{Validazione}
La validazione risulta essere uno dei punti cruciali del progetto. Questo perché permette di verificare che siano stati rispettati gli standard W3C per quanto riguarda HTML e CSS, permettendo così di avere del codice corretto.\\
Codice corretto e validato conferma la solidità di un sito web sui diversi browser. \\
Per validare il sito sono stati utilizzati i seguenti strumenti:
\begin{itemize}
	\item \textbf{W3C HTML Validator}: è un tool online che, dato in input un file .html o del codice HTML, effettua la validazione del codice HTML al fine di assicurare la qualità della pagina.
	\item \textbf{W3C CSS Validator}: è un tool online che, dato in input un file .css o del codice CSS, effettua la validazione del codice CSS, permettendo così di assicurare la validità del codice sottoposto.
\end{itemize}

La Validazione dei file CSS ci ha restituito una corretta validazione, ma con un numero molto alto di avvisi. \\
Questi avvisi sono dati nella quasi totalità da -ms-grid che viene considerata una \textit{vendor extension}. Qeste, come spiegato in precedenza, vengono utilizzate per compatibilità con Internet Explorer.