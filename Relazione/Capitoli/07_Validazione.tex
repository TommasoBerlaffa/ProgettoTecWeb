\section{Validazione}
La validazione risulta essere uno dei punti cruciali del progetto. Questo perché permette di verificare che siano stati rispettati gli standard W3C per quanto riguarda HTML e CSS, permettendo così di avere del codice corretto.\\
Codice corretto e validato conferma la solidità di un sito web sui diversi browser. \\
Per validare il sito sono stati utilizzati i seguenti strumenti:
\begin{itemize}
	\item \textbf{W3C HTML Validator}: è un tool online che, dato in input un file .html o del codice HTML, effettua la validazione del codice HTML al fine di assicurare la qualità della pagina.
	\item \textbf{W3C CSS Validator}: è un tool online che, dato in input un file .css o del codice CSS, effettua la validazione del codice CSS, permettendo così di assicurare la validità del codice sottoposto.
\end{itemize}

La validazione dei file CSS è stata un successo per tutti e tre i file forniti. \\
La validazione dei file HTML ha riscontrato principalmente 2 problemi :
\begin{itemize}
  \item Bad value ... for attribute src on element img: Backslash $("\backslash")$ used as path segment delimiter., \\ questo è dovuto ai Path Delimiter utilizzati in PHP al fine di avere dei delimitatori di path che funzionino sia su Windows che su Linux; Questo errore sarà presente solamente su Windows e non su macchine Linux, come quella in laboratorio;
  \item su signUp.html, è presente un'immagine nel quarto fieldset che verrà caricata tramite javascript, fatto per cui non avrà un src al momento del loading della pagina.
\end{itemize}