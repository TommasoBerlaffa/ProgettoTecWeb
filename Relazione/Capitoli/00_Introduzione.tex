\section{Introduzione}
	\subsection{Abstract}
	Il sito \textbf{Job Finder} si pone come intermediario tra utenti datori di lavoro e utenti lavoratori.\\
  I tipi di lavoro svolti su questo sito saranno esclusivamente e totalmente di tipo \textit{Online}. \\
	Gli utenti dovranno registrarsi per ottenere diversi privilegi, quali cercare lavoro o creare offerte di lavoro, a cui altri utenti potranno rispondere tramite una offerta, 
  chiamata \textit{bid}.\\
  Ogni utente, dopo aver effettuato il login, avrà accesso ad un'area riservata, nella quale potrà visualizzare e modificare i propri dati personali. In quest'area sarò inoltre 
  possibile visualizzare le proprie offerte $($passate e correnti$)$, i lavori svolti e le offerte a cui si sta attualmente candidando.\\
  Non esisterà una divisione tra utenti che offrono lavoro e utenti che si propongono, ma saranno presenti solamente utenti loggati ed Admin. Abbiamo optato per questa scelta poiché 
  uno stesso utente si potrebbe trovare a svolgere un lavoro, ma commissionandone un altro allo stesso tempo.\\
  L'intero sito utilizzerà la lingua inglese come unica lingua disponibile, poiché limitare le possibili offerte di lavoro in base a lingua o nazionalità limiterebbe solamente 
  la possibilità degli utenti di trovare lavoro o un lavoratore.\\
  Viene garantita l’accessibilità al sito, rendendo il sito navigabile a chiunque.\\
  L’usabilità risulta essere un altro tema di particolare importanza: vengono infatti rispettate le linee guida fornite da W3C e viene fatto in modo che la navigazione sul sito
  risulti essere la più fluida possibile, fornendo sempre aiuti per fare in modo che l’utente non si senta perso e sappia dov'é e dove può andare.
	