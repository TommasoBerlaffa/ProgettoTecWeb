\section{Introduzione}
	\subsection{Abstract}
	Il sito Job Finder si pone come intermediario tra datori di lavori e utenti lavoratori, i quali vogliono sfruttare le proprie competenze per trovare un impiego. 
	\\I tipi di lavoro svolti su questo sito sono totalmente di tipo Online. 
	Gli utenti dovranno registrarsi per ottenere diversi privilegi, quali cercare lavoro o creare varie offerte di lavoro, a cui altri utenti potranno rispondere. 
	\\Ogni utente loggato avrà inoltre accesso ad un’area riservata, nella quale potrà visualizzare e modificare i propri dati personali e visualizzare uno storico delle proprie offerte e lavori svolti.
	\\Non esiste una divisione tra Utenti Datori di Lavoro e Utenti Lavoratori, esistono solo Utenti normali ed Admin. Questo scelta è stata fatta poiché uno stesso utente si potrebbe trovare a svolgere un lavoro, ma commissionandone un altro allo stesso tempo.
	\\Viene garantita l’accessibilità al sito, rendendo il sito navigabile a chiunque.
	\\L’usabilità risulta essere un altro tema di particolare importanza : vengono infatti rispettate le linee guida fornite da W3C e viene fatto in modo che la navigazione sul sito risulti essere il più fluida possibile, fornendo sempre aiuti per fare in modo che l’utente non si senta perso.
	