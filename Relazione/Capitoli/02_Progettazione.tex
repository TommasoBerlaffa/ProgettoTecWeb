\section{Progettazione}
	
  \subsection{Design}
  Per il design del sito, sono state perseguite due principali strade :
  \begin{itemize}
    \item \textit{Desktop First}: il nostro sito si rivolge ad un utenza molto tecnica e pensiamo che l'utilizzo del sito avverrà principalmente da Desktop. Viene comunque data importanza all'utenza che deciderà di approcciarsi a questo sito tramite smartphone:
    \item \textit{Layout a 4 Pannelli}: utilizzato in tutte le pagine del sito tranne UserProfile, dove viene utilizzato un Layout a 5 Pannelli,aggiungendo una Sidebar per la navigazione tra gli elementi di userprofile.
  \end{itemize}
  
  I 4 pannelli che compongono il layout delle nostre pagine sono:
  \begin{enumerate}
    \item \textbf{Header}: composto dal logo del sito e dai link alle altre pagine;
    \item \textbf{Breadcrumb}: contiene i Link alle pagine "precedenti" a quella che si sta attualmente visitando; il breadcrumb è di vitale importanza per far sì che l'utente
      non si senta perso nella navigazione del sito. Nel caso del nostro sito, il breadcrump è stato scritto in maniera da simulare un path di un filesystem;
    \item \textbf{Main}: corrisponde al vero e proprio contenuto della pagina;
    \item \textbf{Footer}: contiene informazione su come contattare l'Admin in caso di problemi e il link per le FAQ.
  \end{enumerate}

  Il quinto pannello, overro la \textbf{sidebar}, è utilizzato nella pagina UserProfile. 
  Questa pagina raccoglie informazioni utili dell'utente (informazione su utente, sui lavori svolti e proposti) e la possibilità di modificare i dati utente.


  \subsection{Database}




  \subsection{Obiettivi}
  