\section{Progettazione}
	
  \subsection{Design}
  Per il design del sito, sono state perseguite due principali strade:
  \begin{itemize}
    \item \textit{Desktop First}: il nostro sito si rivolge ad un utenza molto tecnica e pensiamo che l'utilizzo del sito avverrà principalmente da Desktop. Viene comunque data importanza all'utenza che deciderà di approcciarsi a questo sito tramite smartphone:
    \item \textit{Layout a 4 Pannelli}: utilizzato in tutte le pagine del sito tranne UserProfile, dove viene utilizzato un Layout a 5 Pannelli,aggiungendo una Sidebar per la navigazione tra gli elementi di userprofile.
  \end{itemize}
  
  I 4 pannelli che compongono il layout delle nostre pagine sono:
  \begin{enumerate}
    \item \textbf{Header}: composto dal logo del sito e dai link alle altre pagine;
    \item \textbf{Breadcrumb}: contiene i Link alle pagine "precedenti" a quella che si sta attualmente visitando; il breadcrumb è di vitale importanza per far sì che l'utente
      non si senta perso nella navigazione del sito. Nel caso del nostro sito, il breadcrump è stato scritto in maniera da simulare un path di un filesystem;
    \item \textbf{Main}: corrisponde al vero e proprio contenuto della pagina;
    \item \textbf{Footer}: contiene informazione su come contattare l'Admin in caso di problemi e il link per le FAQ.
  \end{enumerate}

  Il quinto pannello, overro la \textbf{sidebar}, è utilizzato nella pagina UserProfile. 
  Questa pagina raccoglie informazioni utili dell'utente $($informazione su utente, sui lavori svolti e proposti$)$, la possibilità di modificare i propri dati utente ed è accessibile solo dallo stesso utente.


  \subsection{Database}




  \subsection{Obiettivi}
  Al fine di migliorare il più possibile l'esperienza utente, il gruppo si è prefissato alcuni obiettivi da perseguire durante la realizzazione di questo progetto :
  \begin{itemize}
    \item utilizzo di tag \textit{alt} per tutte le $<$img$>$ presenti nel sito. Questo non è utile solamente quando l'$<$img$>$ non è disponibile, ma inoltre è cioé che viene letto dagli screen reader, quindi elemento necessario se si vuole fornire la possibilità ad un utente che utilizza questo strumento di comprendere il contesto o il significato di un'immagine;
    \item reinserimento automatico degli input inseriti dall'utente quando avviene il reindirizzamento della pagina sulla stessa 
     $($es: il submit della form di createJob richiama createJob.php che, controllato i dati e confermando un possibile errore, rimanda alla pagina createJob ma ricaricando i valori correttamente inseriti $)$;
    \item uso intelligente del \textit{breadcrumb}, poiché il breadcrumb fornisce informazione all'utente su come é arrivato su quella parte del sito, è fondamentale che il breadrcumb risulti completo;
    \item meccanismi di \textit{hidden help} e \textit{gobacktothetop}, entrambi meccanismi che utilizzano $<$a href="\#"$>$ per condurre l'utente su diversi elementi della pagina. I primi vengono utilizzati da utenti che utilizzano screen reader, i secondi vengono utilizzati quando le pagine risultano verticalmente troppo grandi per tornare "to the top" della pagina $($come suggerisce il nome$)$;
    \item aiuti utente e indicazione sulla correzioni di errori nell'inserimento dati. Nel caso in cui avvengano degli errori, quali mancanza di input o altre tipologie di errori, verranno riportati i cambiamenti da attuare ed il motivo dell'errore, in modo da aiutare l'utente a inserire i dati correttamente.
  \end{itemize} 