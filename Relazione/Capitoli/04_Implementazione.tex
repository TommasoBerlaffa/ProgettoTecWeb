\section{Implementazione}
	

  \subsection{HTML}
  Per questo progetto è stato deciso di utilizzare HTML5 per gestire la struttura. \\
  In ogni pagina HTML saranno presenti i tag meta, che andranno ad indicare diverse informazioni come:
  \begin{itemize}
    \item autori del file;
    \item titolo e descrizione della pagina;
    \item keyword utilizzate per la ricerca della pagina sul web.
  \end{itemize}
  Nel nostro caso, le parole utilizzate come keyword sono il titolo del sito e delle parole che hanno a che fare con lo scopo della pagina, come nel caso di findJob in cui le parole utilizzate saranno "Find a Job Offer".
  I file HTML si andranno a dividere in 2 categorie :
  \begin{itemize}
    \item Pagine HTML, che vengono caricate da PHP e a cui vengono modificati alcuni valori;
    \item Elementi HTML, ovvero file HTML che contengono solo alcuni elementi che poi verranno modificati da PHP e caricati sull'effettiva pagina HTML;
  \end{itemize}

  
  \subsection{CSS}
  Per la parte di presentazione, sono presenti 3 file:
  \begin{itemize}
    \item style.css: per i dispositivi desktop;
    \item mobile.css: per i dispositivi mobile;
    \item print.css: per le funzioni di stampa.
  \end{itemize}
  Viene garantita la separazione tra struttura e presentazione. Di conseguenza, non vengono dichiarati tag style nei file html e si utilizzano file esterni piuttosto che css inline o embeded.
  Per la definizione di dimensioni all'intero del sito, si è optato nella maggior parte dei casi per unità misura come gli em o la percentuale. Alcuni elementi invece utilizzano px, ma si tratta quasi esclusivamente di border.\\
  Viene fatto largo uso di display flex per allieare elementi orizzontalmente o verticalmente in modo da migliorare l'impaginazione di elementi su diverse tipologie di schermi. \\
  Vengono inoltre utilizzati display grid per diversi elementi, come le form di cambio dati utente e le informazioni relative all'offerta di lavoro su FindJob.
  
  \subsection{JavaScript}
  Javascript è uno dei linguaggio utilizzati per il comportamento ed è stato utilizzato per 3 funzionalità:
  \begin{itemize}
    \item controlli su form, per fare in modo che la pagina non dovesse venire ricaricata nel caso ci fossero errori facilmente risolvibili. Alcuni esempi sono il controllo della password, per confermare se soddisfa i requisiti minimi di complessità;
    \item filtraggio tramite tag per findjob e ricerca ed inserimento tag per createJob e area privata;
    \item ricerca tramite input testuale per la zona riservata agli admin.
  \end{itemize}
  JavaScript risulterà essere necessario per alcune funzionalità presenti nel sito, in particolare quella di inserimento dei tag, utilizzate nella pagina di SignUp e di FindJob. Nel caso in cui un utente tenti di utilizzare queste pagine senza 
  JavaScript attivo, viene gentilmente chiesto di attivare JavaScript per usufruire al meglio della pagina.
  Viene inoltre fatto utilizzo di \textbf{AJAX} per ottenere i tag dal server senza dover aggiornare la pagina.

  \subsection{PHP}
  Ogni pagina nel nostro sito viene caricata tramite pagina PHP, questo perché alcuni elementi presenti nell'header verranno visualizzati in maniera diversa in base allo stato di login del'utente. \\
  I file PHP saranno quindi di 3 tipologie:
  \begin{itemize}
    \item file PHP che caricano le pagine HTML e modificheranno alcuni valori, mantenendo così la divisione tra comporamento e struttura poiché quest'ultima sarà descritta nel file .html;
    \item file PHP DBAccess.php, utilizzato per l'interazione con il database, contiene tutti i metodi utilizzati per comunicare con il Database ed ottenere o modificare dati;
    \item file PHP che effettuano azione, al termine delle quali portano ad un'altra pagina.
  \end{itemize}

  Vengono eseguiti controlli sull'input, quali:
  \begin{itemize}
    \item filter\textunderscore var, data una variabile ed il tipo di filter effettua delle operazioni per ottenere una stringa filtrata;
    \item trim, rimuove spazi vuoti all'inizio e alla fine di una stringa.
  \end{itemize}

  \subsection{MySql}
  Per la gestione ed il mantenimento dei dati si é scelto di utilizzare MySql. Viene utilizzata l'estensione misqli per tutte le funzioni che usano MySql. \\
  Si utilizza bind\textunderscore parameters per evitare query injection.