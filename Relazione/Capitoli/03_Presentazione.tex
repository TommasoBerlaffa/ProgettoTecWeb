\section{Presentazione}

  \subsection{Desktop}
    Come dimensione massima per la versione desktop si è deciso di utilizzare 1024px, al fine di evitare difficoltà da parte dell'utente a visualizzare la pagina nella sua interezza. \\
    È presente un punto di rottura a 959px, al fine di migliorare la visibilità e le dimensioni di alcuni elementi.

  \subsection{Mobile}
    In modo simile alla versione Desktop, per la visualizzazione lato Mobile si è optati per punti di rottura, che permettono di gestire al meglio la visualizzazione della pagina. \\
    Il principale punto di rottura per questa versione è a 694px. \\
    La visualizzazione Mobile differisce principalmente per :
    \begin{itemize}
      \item il footer viene rimpiazzato con l'header e viceversa. In questo modo, sarà più semplice per l'utente da mobile utilizzare i link dell'header che fungono come bottoni di un app;
      \item il logo viene rimosso;
      \item i contenuti che precedentemente venivano visualizzati di lato, ora vengono visualizzati in verticale per migliorare l'esperienza utente;
    \end{itemize}
  \subsection{Print}
    